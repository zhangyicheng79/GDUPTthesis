% !TeX root = ../thuthesis-example.tex

\chapter{引用文献的标注}

模板支持 BibTeX 和 BibLaTeX 两种方式处理参考文献。
下文主要介绍 BibTeX 配合 \pkg{natbib} 宏包的主要使用方法。


\section{顺序编码制}

在顺序编码制下,默认的 \cs{cite} 命令同 \cs{citep} 一样,序号置于方括号中,
引文页码会放在括号外。
统一处引用的连续序号会自动用短横线连接。

% \gduptsetup{
%   cite-style = super,
% }
% \bibstyle{super}
\begin{tabular}{l@{\quad$\Rightarrow$\quad}l}
  \verb|\cite{zhangkun1994}|               & \cite{zhangkun1994}               \\
  \verb|\citet{zhangkun1994}|              & \citet{zhangkun1994}              \\
  \verb|\citep{zhangkun1994}|              & \citep{zhangkun1994}              \\
  \verb|\cite[42]{zhangkun1994}|           & \cite[42]{zhangkun1994}           \\
  \verb|\cite{zhangkun1994,zhukezhen1973}| & \cite{zhangkun1994,zhukezhen1973} \\
\end{tabular}


也可以取消上标格式,将数字序号作为文字的一部分。
建议全文统一使用相同的格式。

% \gduptsetup{
%   cite-style = inline,
% }

% \bibstyle{inline}
\begin{tabular}{l@{\quad$\Rightarrow$\quad}l}
  \verb|\cite{zhangkun1994}|               & \cite{zhangkun1994}               \\
  \verb|\citet{zhangkun1994}|              & \citet{zhangkun1994}              \\
  \verb|\citep{zhangkun1994}|              & \citep{zhangkun1994}              \\
  \verb|\cite[42]{zhangkun1994}|           & \cite[42]{zhangkun1994}           \\
  \verb|\cite{zhangkun1994,zhukezhen1973}| & \cite{zhangkun1994,zhukezhen1973} \\
\end{tabular}



\section{著者-出版年制}

著者-出版年制下的 \cs{cite} 跟 \cs{citet} 一样。

% \gduptsetup{
%   cite-style = author-year,
% }

%\bibstyle{author-year}
\begin{tabular}{l@{\quad$\Rightarrow$\quad}l}
  \verb|\cite{zhangkun1994}|                & \cite{zhangkun1994}                \\
  \verb|\citet{zhangkun1994}|               & \citet{zhangkun1994}               \\
  \verb|\citep{zhangkun1994}|               & \citep{zhangkun1994}               \\
  \verb|\cite[42]{zhangkun1994}|            & \cite[42]{zhangkun1994}            \\
  \verb|\citep{zhangkun1994,zhukezhen1973}| & \citep{zhangkun1994,zhukezhen1973} \\
\end{tabular}

\vskip 2ex
% \gduptsetup{
%   cite-style = super,
% }

% \bibstyle{super}
\section{注意事项} (1) 引文参考文献的每条都要在正文中标注\\
\indent (2) 在bib中,中文文献请加上属性 language   = \{ c \}   或  language   = \{ chinese \}  ,\\
\indent (3) BibTeX比较不稳定,当文献引用标记全部变成? 时,请先检查引用标签是否正确,然后再检查Tex文档部分是否存在语法问题,排除Tex部分语法问题后再重新用Xelatex--> BibTeX-->Xelatex*2 编译。
\\
\par
示例:\par
I.\Cspace 连续引用示例:\cite{zhangkun1994,zhukezhen1973} 

II.\space 多篇新文献的连续引用示例:\cite{zhengkaiqing1987, aaas1883science, baishunong1998zhiwu, bixon1996dynamics}

III. \space  引用不存在的标签:  \cite{wronglabelid}

