% ************** Thesis Information & Meta-data ***************
%%=============作者信息=============%
\stunum{202102180102}
\crest{\includegraphics[width=97mm]{imgs/SchoolName}}
\title{\underline{广东石油化工学院 \LaTeX 本科毕设模板} \par \underline{论文中文题目第二行}}
\entitle{\underline{A \LaTeX  ~template for GDUPT undergraduates} \par \underline{the second line of English Title}}
%% Subtitle (Optional)
% \subtitle{Using the CUED template}
\author{张~三}
\advisor{张~翼~成(讲师)}
\school{自动化}%{自动化学院}
\dept{电气工程及其自动化}%{智能科学与技术}
\stucls{自动化19-1}
\designtime{\underline{\makebox[20mm][c]{2021}}年\underline{\makebox[9mm][c]{3}}月\underline{\makebox[9mm][c]{3}}日至\underline{\makebox[20mm][c]{2021}}年\underline{\makebox[9mm][c]{10}}月\underline{\makebox[9mm][c]{20}}日}
%% University 
\university{ 广东石油化工学院 }
% Crest minimum should be 30mm.
%% Submission date
% Default is set as {\monthname[\the\month]\space\the\year}
%\degreedate{September 2014} 

%% Meta information
% \subject{LaTeX} 
% \cnkeywords{{关键词一} {关键词二} {关键词三} {关键词四} {关键词五}}
% \enkeywords{{Keyword01} {Keyword02} {Keyword03} {Keyword04} {Keyword05}}


%%=============其他需要的包=============%%

%======= page 页面设置 =============%
\usepackage[top=2.8cm, bottom=2.2cm, outer=2.2cm, inner=2.8cm]{geometry}

% \setCJKmainfont[...]{...}
% ******************************************************************************
% ******************* Fonts (like different typewriter fonts etc.)*************
% Add `customfont' in the document class option to use this section
% \ifsetCustomFont
%   % Set your custom font here and use `customfont' in options. Leave empty to
%   % load computer modern font (default LaTeX font).
%   %\RequirePackage{helvet}

%   % For use with XeLaTeX
%   %  \setmainfont[
%   %    Path              = ./libertine/opentype/,
%   %    Extension         = .otf,
%   %    UprightFont = LinLibertine_R,
%   %    BoldFont = LinLibertine_RZ, % Linux Libertine O Regular Semibold
%   %    ItalicFont = LinLibertine_RI,
%   %    BoldItalicFont = LinLibertine_RZI, % Linux Libertine O Regular Semibold Italic
%   %  ]
%   %  {libertine}
%   %  % load font from system font
%   %  \newfontfamily\libertinesystemfont{Linux Libertine O}
% \fi

% *****************************************************************************
% **************************** Custom Packages ********************************

% ************************* Algorithms and Pseudocode **************************

%\usepackage{algpseudocode}


% ********************Captions and Hyperreferencing / URL **********************

% Captions: This makes captions of figures use a boldfaced small font.
%\RequirePackage[small,bf]{caption}

% \RequirePackage[labelsep=space, tableposition=top]{caption}
% \renewcommand{\figurename}{Fig.} %to support older versions of captions.sty


% *************************** Graphics and figures *****************************

%\usepackage{rotating}
%\usepackage{wrapfig}

% Uncomment the following two lines to force Latex to place the figure.
% Use [H] when including graphics. Note 'H' instead of 'h'
%\usepackage{float}
%\restylefloat{figure}

% Subcaption package is also available in the sty folder you can use that by
% uncommenting the following line
% This is for people stuck with older versions of texlive
%\usepackage{sty/caption/subcaption}
\usepackage{subcaption}

% ********************************** Tables ************************************
\usepackage{booktabs} % For professional looking tables
\usepackage{multirow}

%\usepackage{multicol}
%长表格
\usepackage{longtable}
\usepackage{threeparttable}
%\usepackage{tabularx}


% *********************************** SI Units *********************************
\usepackage{siunitx} % use this package module for SI units


% ******************************* Line Spacing *********************************

% Choose linespacing as appropriate. Default is one-half line spacing as per the
% University guidelines

% \doublespacing
% \onehalfspacing
% \singlespacing


% ************************ Formatting / Footnote *******************************

% Don't break enumeration (etc.) across pages in an ugly manner (default 10000)
%\clubpenalty=500
%\widowpenalty=500

%\usepackage[perpage]{footmisc} %Range of footnote options


% *****************************************************************************
% *************************** Bibliography  and References ********************

%\usepackage{cleveref} %Referencing without need to explicitly state fig /table

% Add `custombib' in the document class option to use this section

%  \RequirePackage[square, sort, numbers, authoryear]{natbib} % CustomBib

% If you would like to use biblatex for your reference management, as opposed to the default `natbibpackage` pass the option `custombib` in the document class. Comment out the previous line to make sure you don't load the natbib package. Uncomment the following lines and specify the location of references.bib file

%\RequirePackage[backend=biber, style=numeric-comp, citestyle=numeric, sorting=nty, natbib=true]{biblatex}
%\addbibresource{References/references} %Location of references.bib only for biblatex, Do not omit the .bib extension from the filename.
\usepackage{gbt7714}

% changes the default name `Bibliography` -> `References'
\renewcommand{\bibname}{\Large 参考文献}
\renewcommand{\contentsname}{\Large 目\Cspace 录}
\renewcommand{\listfigurename}{\Large 插\Cspace 图}
\renewcommand{\listtablename}{\Large 表\Cspace 格}
\renewcommand{\abstractname}{摘\Cspace 要}

% ******************************************************************************
% ************************* User Defined Commands ******************************
% ******************************************************************************

% *********** To change the name of Table of Contents / LOF and LOT ************

%\renewcommand{\contentsname}{My Table of Contents}
%\renewcommand{\listfigurename}{My List of Figures}
%\renewcommand{\listtablename}{My List of Tables}


% ********************** TOC depth and numbering depth *************************

\setcounter{secnumdepth}{2}
\setcounter{tocdepth}{2}


% ******************************* Nomenclature *********************************

% To change the name of the Nomenclature section, uncomment the following line

%\renewcommand{\nomname}{Symbols}


% ********************************* Appendix ***********************************

% The default value of both \appendixtocname and \appendixpagename is `Appendices'. These names can all be changed via:

%\renewcommand{\appendixtocname}{List of appendices}
%\renewcommand{\appendixname}{Appndx}

% *********************** Configure Draft Mode **********************************

% Uncomment to disable figures in `draft'
%\setkeys{Gin}{draft=true}  % set draft to false to enable figures in `draft'

% These options are active only during the draft mode
% Default text is "Draft"
%\SetDraftText{DRAFT}

% Default Watermark location is top. Location (top/bottom)
%\SetDraftWMPosition{bottom}

% Draft Version - default is v1.0
%\SetDraftVersion{v1.1}

% Draft Text grayscale value (should be between 0-black and 1-white)
% Default value is 0.75
%\SetDraftGrayScale{0.8}


% ******************************** Todo Notes **********************************
%% Uncomment the following lines to have todonotes.

%\ifsetDraft
%	\usepackage[colorinlistoftodos]{todonotes}
%	\newcommand{\mynote}[1]{\todo[author=kks32,size=\small,inline,color=green!40]{#1}}
%\else
%	\newcommand{\mynote}[1]{}
%	\newcommand{\listoftodos}{}
%\fi

% Example todo: \mynote{Hey! I have a note}

% *****************************************************************************
% ******************* Better enumeration my MB*************
\usepackage{enumitem}

% 定义所有的图片文件在 figures 子目录下
\graphicspath{{figs/}}

% % 数学命令
% \newcommand\dif{\mathop{}\!\mathrm{d}}  % 微分符号

% hyperref 宏包在最后调用
\usepackage{hyperref}


